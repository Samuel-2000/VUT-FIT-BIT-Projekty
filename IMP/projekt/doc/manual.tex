\documentclass[12pt,a4paper,final]{article}

\usepackage[czech]{babel}
\usepackage[utf8]{inputenc}

\usepackage[bookmarksopen, colorlinks, plainpages=false, urlcolor=blue, unicode]{hyperref}
\usepackage{url}
\usepackage{amsmath}
\usepackage{capt-of}
\usepackage[Q=yes]{examplep}
\usepackage{enumitem}
\usepackage{graphicx}
\usepackage[text={17.cm, 25cm}, ignorefoot]{geometry}
\usepackage{ragged2e}
\justifying

\begin{document}

\begin{titlepage}

\vspace*{2cm}
\begin{figure}[!h]
  \centering
  \includegraphics[width=13cm]{vut-fit-logo.pdf}
\end{figure}

\vfill

\begin{center}

\begin{Huge}
Aplikace modulu Watchdog Timer (WDOG) 
\end{Huge}

\bigskip

\begin{Large}
Mikroprocesorové a vstavané systémy
\end{Large}

\bigskip

\begin{Large}
	2022/2023
\end{Large}

\end{center}

\vfill

\begin{center}
\begin{Large}
\today
\end{Large}
\end{center}

\vfill

\begin{flushleft}
\begin{normalsize}
\begin{tabular}{ll}
\bf Autor: & Samuel Kuchta (xkucht11)
\end{tabular}
\end{normalsize}
\end{flushleft}
\end{titlepage}


% Obsah
\pagestyle{plain}
\pagenumbering{roman}
\setcounter{page}{1}
\tableofcontents


\newpage
\pagestyle{plain}
\pagenumbering{arabic}
\setcounter{page}{1}

\section{Zadání}
Zadáním projektu bylo vytvořit aplikaci demonstrující možnosti modulu Watchdog Timer (WDOG) dostupného na mikrokontroléru Kinetis K60 na platformě FITkit 3.

Předpoklady projektu:
\begin{itemize}
    \item Předpokládejte zdroj hodin LPO (Low-Power Oscillator).
    \item V rámci jednoduché vestavné aplikace demonstrujte včasnou obsluhu WDOG v periodickém a okénkovém (windowed) režimu, každou s různými velikostmi periody/okna.
    \item Aplikace musí interagovat (např. pomocí tlačítek, LED, piezzo bzučáku či terminálu) s uživatelem; přinejmenším musí (rozlišitelným způsobem) signalizovat pomocí LED a/nebo piezzo bzučáku: zahájení vestavné aplikace po resetu, zahájení iterace cyklu v aplikaci, zahájení obsluhy WDOG.
    \item Pomocí aplikace musí být možno demonstrovat dopad nevčasných obsluh WDOG na její chod a provést sběr statistik o počtech a příčinách resetu mikrokontroléru.
\end{itemize}

\section{Watchdog}

Watchdog je časovač, který je používán ke kontrole a detekci zacyklení aplikací. Při běžném provozu je watchdog pravidelně resetován běžící aplikací, což zabraňuje vypršení jeho časového limitu. Pokud běžící aplikace nedokáže kvůli nějaké chybě resetovat watchdog, časovač uplyne a systém je zaveden resetem do původního stavu.

\subsection{Registry modulu WDOG (16-bit)}
\begin{itemize}
    \item \verb|WDOG_STCTRLH| a \verb|WDOG_STCTRLL|: slouží ke konfiguraci.
    \begin{itemize}
        \item Zapnutí Watchdogu
        \item Zapnutí/vypnutí okénkového režimu
        \item Nastavení zdroje hodin (LPO)
        \item Povolení v režimu CPU Debug.
    \end{itemize}
    \item \verb|WDOG_TOVALH| a \verb|WDOG_TOVALL|: Nastavení velikosti horní periody časovače.
    \item \verb|WDOG_WINH| a \verb|WDOG_WINL|: Nastavení velikosti spodní periody časovače. (Jen v okénkovém režimu).
    \item \verb|WDOG_REFRESH|: zápisem 2 specifických hodnot se resetuje horní perioda časovače, co oddálí reset programu watchdogem.
    \item \verb|WDOG_UNLOCK|: zápisem 2 specifických hodnot povolí změny hodnot určitých registrů. 
    \item \verb|WDOG_TMROUTH| a \verb|WDOG_TMROUTL|: obsahují uplynulý čas od posledního obnovení časovače.
    \item \verb|WDOG_RSTCNT|: počet resetů MCU.
    \item \verb|WDOG_PRESC|: Dělička vstupních hodin, určená pro zpomalení taktu.
\end{itemize}

\subsection{Periodický režim}
Využíván pro kontrolu správného běhu programu (jestli nedošlo k zacyklení). Jestli po dobu periody (zapsané v registrech \verb|WDOG_TOVALH| a \verb|WDOG_TOVALL|) nebude WDOG obnoven, systém bude resetován.

\subsection{Okénkový režim}
Využíván pro kontrolu správnosti taktu hodin (věkem se může takt hodin změnit, co by mohlo ohrozit některé aplikace). Horní perioda se nastavuje stejně jako v periodickém režimu. Spodní perioda se nastavuje v registrech \verb|WDOG_WINH| a \verb|WDOG_WINL|. Pokud nastane obnovení watchdogu mimo tento interval, systém bude resetován.


\section{Implementace}

\begin{itemize}
    \item Jazyk: C.
    \item IDE: Kinetis Desgin Studio.
    \item Zdrojový soubor: \verb|main.c|
\end{itemize}

\subsection{Uživatelské rozhraní}
Program komunikuje s uživatelem prostřednictvím bzučáku, LED, a výpisů do terminálu.

\subsection{Tok programu}
\begin{itemize}
    \item Inicializuje se MCU, Porty a UART.
    \item Při každém resetu nastane krátke zapípání.
    \item Inicializuje se WDOG, vybere se režim podle počtu resetů. Jestli je počet resetů sudý, tak je vybrán periodický režim, v opačném případě je vybrán okénkový režim.
    \item Při prvním startu (Power-On Reset) zahraje znělku Windows XP Start-up.
    \item Začne hrát skladbu "Pro Elišku"  od Beethovena.
    \item Periodický režim:
    \begin{itemize}
        \item Po prvním dohrání skladby se automaticky obnoví WDOG
        \item Při druhém přehrání skladby nastane zacyklení programu (LED Signalizace)
        \item Po uplynutí časovače je prográm resetován.
    \end{itemize}
    \item Okénkový režim:
    \begin{itemize}
        \item Po prvním dohrání skladby má možnost uživatel obnovit časovač stlačením tlačítka SW6. Jestli tak neučiní, vyprší časovač, a program je resetován.
        \item Nastane brzký automatický refresh, který simuluje nežádoucí zrychlení taktu hodin MCU. (Hodnota bude menší než je nastaveno v Registrech dolní periody, čímž dojde k resetu.)
    \end{itemize} 
\end{itemize}

\subsection{Výpis statistik}
Do terminálu jsou zasílány informace o pořadí resetu, jeho příčině, a o právě používaném režimu.

\section{Použití}
Aplikaci je vhodné spustit s terminálem (např. PuTTY), který bude sloužit k výpisu informací o stavu aplikace.

\subsection{Nastavení Sériového rozhraní}
Správce zařízení a Putty:
\begin{itemize}
    \item Speed (baud): 115200
    \item Data bits: 8
    \item Stop bits: 1
    \item Parity: None
    \item Flow Control: None
\end{itemize} 

\section{Shrnutí}
Aplikace Používá modul Watchdog pro resetování programu při jeho nesprávném fungování.

\end{document}